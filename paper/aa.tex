%-----------------------------------------------------------------------
%                                                                 aa.tex
% AA vers. 9.3, LaTeX class for Astronomy & Astrophysics
%-----------------------------------------------------------------------

\documentclass{aa}  

\usepackage{graphicx}
\usepackage{txfonts}
\usepackage{lipsum}
\usepackage{subcaption}
\usepackage{lscape}
\usepackage{placeins}

% ==== 这里新加:超链接包,放在所有包之后 ====
\usepackage[colorlinks=true, allcolors=blue]{hyperref}
% ==============================================

\begin{document}

\title{Predicting NBA Team Performance: \\
From Historical Standings to Player-level Forecasts}

\author{
    Weihao Li\inst{1}\fnmsep\thanks{Corresponding author: wli@u.northwestern.edu}
    \and Shuhao Gao\inst{1}
    \and C. Brassac\inst{1}
    \and C. Estievenart\inst{1}
}

\institute{
    McCormick School of Engineering, Northwestern University \\
    \email{\{weihaoli2027, shuhaogao2027, author3, author4\}@u.northwestern.edu}
}

\date{Received December 8, 2025}

%-----------------------------------------------------------------
% ABSTRACT
%-----------------------------------------------------------------
\abstract
{In recent years, publicly available NBA play-by-play and box-score data have made it possible to build increasingly detailed models of team performance. At the same time, many forecasting systems still rely on simple team-level summaries such as past wins or point differential.}
{We aim to compare a simple standings-based baseline with a player-centric forecasting pipeline for predicting future regular-season wins at the team level.}
{We construct two models. The first directly regresses future team wins on historical win totals over the past two decades. The second learns generic player evolution patterns from a long-horizon panel of both retired and active players, uses these patterns to predict next-season box-score statistics for active players, and aggregates player-level projections into team-level features.}
{Across held-out seasons, the player-based pipeline improves mean absolute error relative to the standings baseline, especially for teams undergoing large off-season roster changes. For more stable teams, historical standings are surprisingly competitive and sometimes match the player-based forecasts.}
{Our results suggest that simple team-level baselines remain strong when rosters are stable, but player-level modeling adds value in high-turnover regimes and provides a more interpretable connection between roster moves and expected team performance.}

\keywords{basketball --
          sports analytics --
          machine learning --
          time series forecasting --
          player development}

\maketitle

%%%%%%%%%%%%%%%%%%%%%%%%%%%%%%%%%%%%%%%%%%%%%%%%%%%%%%%%%%%%%%
\section{Introduction}

In this project, we study how well different approaches can predict NBA team performance. Our goal is straightforward: given past data, can we forecast how many games each team will win in a future season? To explore this question, we compare two models that rely on fundamentally different sources of information.

The first model is a simple team-level baseline that uses only historical standings. For any pair of seasons within the past twenty years, we examine how a team's previous win total relates to its win total in another season. This provides a direct, standings-based method for predicting future performance without considering player-level factors.

The second model takes a player-centered perspective. We collect long-term statistics for both retired and still-active NBA players and train a model to learn how player performance tends to evolve over time. Using these learned patterns, we predict next-season box-score statistics for current active players and then aggregate the projected player output to estimate each team’s future win total.

By comparing these two approaches, we aim to understand when a simple
standings-based baseline is sufficient and when a more detailed, player-based forecasting pipeline provides an advantage, particularly in seasons affected by significant roster changes.

%%%%%%%%%%%%%%%%%%%%%%%%%%%%%%%%%%%%%%%%%%%%%%%%%%%%%%%%%%%%%%
\section{Prior Literature}

\subsection{Team-level rating and expectation models}

A large body of work in sports analytics has focused on predicting team
performance using aggregate team-level statistics. One influential family
of models is the \emph{Pythagorean expectation}, which estimates a team's
theoretical winning percentage from points scored and points allowed via
a power-law relationship. Originally proposed for baseball and later
adapted to basketball, this idea underlies many simple baselines that
relate scoring margins to wins and losses
\citep[see, e.g.,][]{oliver2004basketball,sarlis2020sports}.
These models provide strong, easy-to-interpret benchmarks but depend
solely on aggregate scoring margins and do not incorporate information
about individual players.

Another common approach is to model team strength with rating systems
inspired by Elo. In basketball applications, Elo-style ratings are updated
after each game based on the result, home-court advantage, and the margin
of victory, and then used to forecast future game and series outcomes.
Public-facing systems such as FiveThirtyEight's NBA model illustrate how
dynamic ratings can track changes in team strength over a season and
provide reasonably accurate probabilistic predictions for both games and
playoff series \citep{silver2015elo}. Together, Pythagorean and Elo-style
systems represent static or quasi-static team-level baselines that are
closely related to our first model, which relies on historical team wins
and standings. However, because these methods operate on aggregate team
outcomes, they are inherently limited in their ability to anticipate
abrupt changes in performance driven by roster turnover or player
development.

\subsection{Machine learning for NBA game and season prediction}

Beyond analytic formulas and rating systems, many studies have applied
machine learning to predict basketball results using team-level features.
Early work by \citet{loeffelholz2009nba} used neural networks to predict
single-game NBA outcomes from box-score statistics and contextual
variables, demonstrating that nonlinear models can capture interactions
between basic team statistics. More recent surveys review a wide range of
approaches, including logistic regression, support vector machines,
tree-based ensembles, and deep neural networks, and typically find that
machine-learning models outperform simpler statistical baselines when
sufficient historical data are available \citep{sarlis2020sports}.

Researchers have also moved from game-level prediction to season-level
tasks, such as forecasting a team's final win total or playoff
qualification. For example, \citet{yang2015nba} regress regular-season
wins on team-level and aggregated player statistics to study which
factors are most predictive of team success. These season-level models
again treat each team as the unit of analysis and usually rely on summary
statistics from the current or previous season.

\subsection{Player-level prediction and its link to team performance}

Complementary to team-level approaches, a growing literature studies
player evaluation and performance prediction using detailed box-score and
tracking data. \citet{sarlis2020sports} review many of these methods,
including regression-based models for player efficiency metrics,
clustering techniques for grouping players with similar playing styles,
and rating systems that quantify individual contribution to team success.
In practice, coaches and analysts often combine such player-level models
with domain knowledge to support decisions about rotations, matchups, and
roster construction.

Some work, including \citet{yang2015nba}, aggregates player statistics
into team-level features to predict season outcomes, effectively creating
a simple player-to-team pipeline. However, existing player-focused models
typically train and evaluate on overlapping time periods for the same set
of players, and they rarely combine long-run player evolution with
explicit team-level baselines.

In contrast, our project is designed to learn generic player evolution
patterns from a historical cohort that includes both retired and
still-active players, and then apply these patterns to predict the
next-season performance of currently active players. We subsequently
aggregate predicted player statistics to the team level and compare the
resulting win forecasts to those from a simple standings-based baseline
over roughly two decades of NBA data. This setup allows us to evaluate
how a player-based forecasting pipeline and a standings-based model
perform side by side, especially in seasons where teams undergo
substantial roster changes.

%%%%%%%%%%%%%%%%%%%%%%%%%%%%%%%%%%%%%%%%%%%%%%%%%%%%%%%%%%%%%%
\section{Example notation and basic formulas}

% ...(这一节、“Figures examples”、“Tables examples” 等你原来的内容不变)...

%%%%%%%%%%%%%%%%%%%%%%%%%%%%%%%%%%%%%%%%%%%%%%%%%%%%%%%%%%%%%%
\section{Conclusions}

Lorem ipsum dolor sit amet, consectetuer adipiscing elit. In hac
habitasse platea dictumst. Integer tempus convallis augue. Etiam
facilisis.

%%%%%%%%%%%%%%%%%%%%%%%%%%%%%%%%%%%%%%%%%%%%%%%%%%%%%%%%%%%%%%
\begin{acknowledgements}
We thank the open basketball-reference and NBA Stats communities for
making historical data accessible, and the Northwestern MSCS program
for general support of this work.
\end{acknowledgements}

% ==== 参考文献:保持 A&A 的 aa.bst,author–year 风格 ====
\bibliographystyle{aa}
\bibliography{refs}

\begin{appendix}
% ... 你的 appendix 内容 ...
\end{appendix}

\end{document}
